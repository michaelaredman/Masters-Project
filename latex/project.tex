
\documentclass{article}
\usepackage{amsmath}
\usepackage{amsfonts}
\usepackage{mathabx}
\usepackage{amsthm}
\usepackage{amssymb}
\usepackage{framed}
\usepackage{relsize}
\usepackage{graphicx}
\graphicspath{ {images/}}
\usepackage{subcaption}
\usepackage{enumerate}
\usepackage{booktabs}
\usepackage[usenames,dvipsnames]{color}
\usepackage{tikz}
\usepackage{listings}
\lstset{language=Python, breaklines=true}  
\usepackage{fancyhdr}
\setlength{\parindent}{0pt}
\pagestyle{fancy}
\fancyhf{}
\fancyhead[R]{Michael Redman, CID:\ 00826863}
\fancyfoot[C]{\thepage}
\begin{document}
\title{Modelling spatiotemporal variance in epidemiological contexts}
\author{Michael Redman, CID:\ 0082686\textbf{3}}
\date{December 2016}

\maketitle

\section{The problem}

The correct identification of blah insert stuff here.

\section{The Model}

Consider

\section{Smoothing}

Ideally we wish to identify potential local risk factors in the aetiology of a disease, say, carcenogenic hazard from industrial polution. So it's clear that the ability to incorporate a high level of spatial granularity in our model is of value in these contexts. However this comes with the trade-off of greater variance in the counts making identification of abnormal temporal trends difficult, especially for diseases with low incidence. Therefore we need to employ an element of smoothing over the neibourhoods of the 




\end{document}
